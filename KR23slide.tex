\documentclass[usenames,dvipsnames]{beamer}
% \linespread{1.3}
\mode<presentation>{
\usetheme{Madrid}
\usecolortheme{default}
%\usecolortheme{beaver}
}
\usepackage{array}
\usepackage[utf8]{inputenc}
\usepackage{utopia}
\usepackage{verbatim}
\usepackage[portuguese]{babel}
\usepackage{pgfplots}
\pgfplotsset{/pgf/number format/use comma,compat=newest}
\usepackage{color}
\usepackage{amsmath,amsfonts,amssymb}
\usepackage{hyperref}
\usepackage{tikz}
\usepackage{proof}
\usepackage{caption}
\usepackage{soul}
%https://www.overleaf.com/project/61717826e515873b24296576
\usepackage{graphicx}
\usepackage{algorithm2e}
\usepackage{algorithmic}
\usepackage{xcolor}
% \usepackage{setspace}

\usetikzlibrary{positioning}

\setlength{\parskip}{1em}

\newcommand\ldiaarg[1]{\langle#1\rangle}

\newcommand{\M}{\mathcal{M}}
\newcommand{\LL}{\mathcal{L}} %for automata

\newcommand{\POL}{\mathsf{POL}}
\newcommand{\EPL}{\mathsf{EPL}}
\newcommand{\Decide}{\mathsf{Decide}}
\newcommand{\DecidePSPACE}{\mathsf{mcPOL}}
\newcommand{\reach}{\mathsf{PSPACEReach}}
\newcommand{\CreateDelta}{\mathsf{CreateDelta}}
\newcommand{\StringRepresent}{\mathsf{StringRepresent}}
\newcommand{\StoreStrings}{\mathsf{StoreStrings}}
\newcommand{\ResidueByLetter}{\mathsf{ResidueByLetter}}
\newcommand{\ResidueByWord}{\mathsf{ResidueByWord}}
\newcommand{\AuxOut}{\mathsf{AuxOut}}
\newcommand{\GetSetNP}{\mathsf{GetSetNP}}
\newcommand{\T}{\mathsf{True}}
\newcommand{\Fa}{\mathsf{False}}
\newcommand{\tr}{\mathsf{tr}}
\newcommand{\starfree}{\mathsf{Star\mbox{-}Free}}
\newcommand{\word}{\mathsf{Word}}
\newcommand{\existential}{\mathsf{Existential}}
\newcommand{\PSPACE}{\mathsf{PSPACE}}
\newcommand{\NPSPACE}{\mathsf{NPSPACE}}
\newcommand{\PTime}{\mathsf{P}}
\newcommand{\NP}{\mathsf{NP}}
\newcommand{\PTIME}{\mathsf{PTIME}}
\newcommand{\automaton}{\mathcal A}
\newcommand{\modelM}{\mathcal M}
\newcommand{\languageof}[1]{L({#1})}
\newcommand{\set}[1]{\{#1\}}
\newcommand{\suchthat}{\mid}
\newcommand{\union}{\cup}
\newcommand{\Union}{\bigcup}
\newcommand{\ep}{\ensuremath{\varepsilon}}

\newcommand{\obsright}{\blacktriangleright}
\newcommand{\obsleft}{\blacktriangleleft}
\newcommand{\obsup}{\blacktriangle}
\newcommand{\obsdown}{\blacktriangledown}

\newcommand{\expwater}{(\obsright \union \obsup)^* (\obsdown \union \obsleft \union \ep) (\obsright \union \obsup)^*}
\newcommand{\exppower}{(\obsleft \union \obsdown)^* (\obsup \union \obsright \union \ep) (\obsleft \union \obsdown)^*}
\newcommand{\exppatrol}{(\obsright^+ \obsdown^+ \obsleft^+ \obsup^+)^*}

\newcommand\drone{\includegraphics[height=3ex]{images/E1D2_color.png}}
\newcommand\agentA{\includegraphics[height=3ex]{images/agent A icon.png}}
\newcommand\agentB{\includegraphics[height=3ex]{images/agent b icon.png}}
\newcommand\water{\includegraphics[height=3ex]{images/1F4A7_color.png}}

\newcommand\algoaccept{\textbf{accept}}
\newcommand\algoreject{\textbf{reject}}
\newcommand{\regdiv}[1]{\ensuremath{\backslash} #1}

\renewcommand{\phi}{\varphi}

\tikzstyle{rectNode} = [rectangle, text centered]
\tikzstyle{fastate} = [circle, text centered, draw = black]
\tikzstyle{dots} = [circle, draw = black, fill = black, inner sep=1pt]

\tikzstyle{world} = [draw]
\tikzstyle{worldwhite} = [draw = white]

\makeatletter
\let\HL\hl
\renewcommand\hl{%
  \let\set@color\beamerorig@set@color
  \let\reset@color\beamerorig@reset@color
  \HL}
\makeatother



\title[]{On Simple Expectations and Observations of Intelligent Agents: A Complexity Study}
\author[]{
Sourav Chakraborty\inst{1}
\and
\textbf{Avijeet Ghosh}\inst{1}\and
Sujata Ghosh\inst{1}\and\\
Fran{\c{c}}ois Schwarzentruber\inst{2}}
\institute[]{\inst{1} Indian Statistical Institute\and \inst{2} Univ Rennes, IRISA}
\date[KR 23, September 2023]

\AtBeginSection[]
{
\begin{frame}{Table of Contents}
\tableofcontents[currentsection]
\end{frame}
}


\begin{document}

\begin{frame}
 \maketitle
\end{frame}
\begin{frame}{Table of Contents}
    \tableofcontents
\end{frame}

\section{Motivation}
\begin{frame}{Motivation:A Farming Drone}
    \begin{figure}
        \centering
        \includegraphics[scale=0.15]{images/drone-water-power.jpg}
    \end{figure}
    \begin{itemize}
        \item A water source on top right corner
        \item A power source on bottom right corner
    \end{itemize}
\end{frame}

% \begin{frame}{A Farming Drone}
%     \begin{figure}
%         \centering
%         \includegraphics[scale=0.15]{images/no-diag.jpg}
%     \end{figure}
%     \begin{itemize}
%         \item Cannot move diagonally
%     \end{itemize}
% \end{frame}

\begin{frame}{Motivation:A Farming Drone}
    \begin{figure}
        \centering
        \includegraphics[scale=0.15]{images/drone-move.jpg}
    \end{figure}
    \begin{itemize}
        \item Can move up, down, left or right
        \item Cannot move diagonally
    \end{itemize}
\end{frame}

% \begin{frame}{Farming Drone: Agents}
%     \begin{figure}
%         \centering
%         \includegraphics[scale=0.2]{images/a-expects.jpg}
%         \includegraphics[scale=0.2]{images/b-expects.jpg}
%     \end{figure}
%     \begin{itemize}
%         \item Two agents: A and B
%     \end{itemize}
% \end{frame}

% \begin{frame}{Farming Drone: Agents}
%     \begin{figure}
%         \centering
%         \includegraphics[scale=0.2]{images/a-expects-move.jpg}
%         \includegraphics[scale=0.2]{images/b-expects.jpg}
%     \end{figure}
%     A expects
%     \begin{itemize}
%         \item Drone moves towards water or power source with at most one error move.
%     \end{itemize}
% \end{frame}

\begin{frame}{Farming Drone: Agents and their expectations}
    \begin{figure}
        \centering
        \includegraphics[scale=0.2]{images/a-expects.jpg}
        \includegraphics[scale=0.2]{images/b-expects.jpg}
    \end{figure}
     \begin{columns} 
% Column 1
    \begin{column}{.5\textwidth}
    \begin{itemize}
        \item[\textcolor{white}{\textbullet}] \textcolor{white}{Go to water with $\leq 1$ wrong move.}
        \item[\textcolor{white}{\textbullet}] \textcolor{white}{Go to power with $\leq 1$ wrong move.}
        \item[\textcolor{white}{\textbullet}] \textcolor{white}{Go patrolling in clockwise direction.}
    \end{itemize}
    \end{column}
% Column 2    
    \begin{column}{.5\textwidth}
        
    \end{column}
    \end{columns}
\end{frame}


\begin{frame}{Farming Drone: Agents and their expectations}
    \begin{figure}
        \centering
        \includegraphics[scale=0.2]{images/a-expects-patrol.jpg}
        \includegraphics[scale=0.2]{images/b-expects.jpg}
    \end{figure}
     \begin{columns} 
% Column 1
    \begin{column}{.5\textwidth}
    \begin{enumerate}
        \item Go to water with $\leq 1$ wrong move.
        \item Go to power with $\leq 1$ wrong move.
        \item Go patrolling in clockwise direction.
    \end{enumerate}
    \end{column}
% Column 2    
    \begin{column}{.5\textwidth}
        
    \end{column}
    \end{columns}
\end{frame}

\begin{frame}{Farming Drone: Agents and their expectations}
    \begin{figure}
        \centering
        \includegraphics[scale=0.2]{images/a-expects-patrol.jpg}
        \includegraphics[scale=0.2]{images/b-expects-move.jpg}
    \end{figure}
    \begin{columns} 
% Column 1
    \begin{column}{.5\textwidth}
    \begin{enumerate}
        \item Go to water with $\leq 1$ wrong move.
        \item Go to power with $\leq 1$ wrong move.
        \item Go patrolling in clockwise direction.
    \end{enumerate}
    \end{column}
% Column 2    
    \begin{column}{.5\textwidth}
    \begin{enumerate}
        \item Go to water with $\leq 1$ wrong move.
        \item Go to power with $\leq 1$ wrong move.
    \end{enumerate}
    \end{column}
    \end{columns}
\end{frame}

\begin{frame}{Farming Drone: Reasoning about this scenario}
    \begin{columns}
    \begin{column}{0.3\textwidth}
    \begin{figure}
        \centering
        \includegraphics[scale=0.15]{images/farming A-towardsWater.jpg}
    \end{figure}
    \end{column}
    \begin{column}{0.7\textwidth}
    \begin{itemize}
        \item<1-> What is the minimal number of \raisebox{-1.5mm}{\drone} moves that A has to \textbf{observe} to know its goal?
        \item<2-> Does there exist a sequence of \raisebox{-1.5mm}{\drone} moves such that by observing it, B would know its goal but A would not?
    \end{itemize}
    \end{column}
    \end{columns}
    
\end{frame}

\section{Introduction: Public Observation Logic ($\POL$)}

\begin{frame}{Epistemic Expectation Model\footnotemark[1]}
    \begin{itemize}
         \item Epistemic model (W, R, V).
        \item[\textcolor{white}{\textbullet}] \textcolor{white}{Each world is assigned with a regular expression: a set of expected observations.}
        % \item[\textcolor{white}{\textbullet}] \textcolor{white}{Each expression is expected observations in that particular possible world.}
        \item[\textcolor{white}{\textbullet}] \textcolor{white}{Model can get truncated after an observation, say $\obsright\obsdown\obsleft$.}
    \end{itemize}
    \begin{figure}
%         \begin{tikzpicture}
% 			\node at (8, 1.5) {$W = \{u, s, t\}$};
% 			\node at (8, 1) {$V: u\leftarrow patrol, s\leftarrow water, t\leftarrow power$};
% 			\node at (8, 0.5) {$R : (u-A-s), (u-A-t), (s-A-t), (s-B-t)$};
% 		\end{tikzpicture}\\
        \pause
        \begin{tikzpicture}[yscale=1.3]
			\node[world] (s) {$water$};
			\node at (-0.5, -0.4) {\textcolor{white}{\small$\expwater$}};
			\node[world] (t) at (4, 0) {$power$};
			\node at (4.5, -0.4) {\textcolor{white}{\small$\exppower$}};
			\node[world] (u) at (0, 1) {$patrol$};
			\node at (2.3, 1) {\textcolor{white}{\small$\exppatrol$}};
			\node[left = 0mm of s] {$s$};
			\node[right = 0mm of t] {$t$};
			\node[left = 0mm of u] {$u$};
			\draw (s) edge node[above] {$A,B$} (t);
			\draw (s) edge node[left] {$A$} (u);
			\draw (t) edge node[above] {$A$} (u);
		\end{tikzpicture}
		
    \end{figure}
    \footnotetext[1]{\tiny H van Ditmarsch, S Ghosh,
R Verbrugge, and Y Wang. Hidden protocols: Modifying our expectations in an evolving world. Artificial Intelligence, 208:18–40, 2014}
\end{frame}

\begin{frame}{Epistemic Expectation Model\footnotemark[1]}
    \begin{itemize}
         \item Epistemic model (W, R, V).
        \item Each world is assigned with a regular expression: a set of expected observations.
        % \item Each expression is expected observations in that particular possible world.
        \item[\textcolor{white}{\textbullet}] \textcolor{white}{Model can get truncated after an observation, say $\obsright\obsdown\obsleft$.}
    \end{itemize}
    \begin{figure}
        \newcommand{\sizefield}{7}
		\begin{tikzpicture}[scale=0.3]
			\foreach \x in {0, 1, ..., \sizefield} {
				\draw (\x, 0) -- (\x, \sizefield);
				\draw (0, \x) -- (\sizefield, \x);
			}
			\node at (0.5, 0.5) {\includegraphics[width=0.3cm]{images/E097_color.png}};%power
			\node at (6.5, 6.5) {\includegraphics[width=0.3cm]{images/1F4A7_color.png}};%water
			\node at (3.5, 3.5) {\includegraphics[width=0.3cm]{images/E1D2_color.png}};%drone
		\end{tikzpicture}
        \begin{tikzpicture}[yscale=1.3]
			\node[world] (s) {$water$};
			\node at (-0.5, -0.4) {\small$\expwater$};
			\node[world] (t) at (4, 0) {$power$};
			\node at (4.5, -0.4) {\small$\exppower$};
			\node[world] (u) at (0, 1) {$patrol$};
			\node at (2.3, 1) {\small$\exppatrol$};
			\node[left = 0mm of s] {$s$};
			\node[right = 0mm of t] {$t$};
			\node[left = 0mm of u] {$u$};
			\draw (s) edge node[above] {$A,B$} (t);
			\draw (s) edge node[left] {$A$} (u);
			\draw (t) edge node[above] {$A$} (u);
% 		\end{tikzpicture}
% % 		\begin{tikzpicture}
% 			\node at (8, 1.5) {$\obsleft$: one step left};
% 			\node at (8, 1) {$\obsright$: one step right};
% 			\node at (8, 0.5) {$\obsup$: one step up};
% 			\node at (8, 0) {$\obsdown$: one step down};
		\end{tikzpicture}
	    
    \end{figure}
    \footnotetext[1]{\tiny H van Ditmarsch, S Ghosh,
R Verbrugge, and Y Wang. Hidden protocols: Modifying
our expectations in an evolving world. Artificial Intelligence, 208:18–40, 2014}
\end{frame}

\begin{frame}{Epistemic Expectation Model\footnotemark[1]}
    \begin{itemize}
         \item Epistemic model (W, R, V).
        \item Each world is assigned with a regular expression: a set of expected observations.
        % \item Each expression is expected observations in that particular possible world.
        \item Model can get truncated after a sequence of observation, say $\obsright$.
    \end{itemize}
    \begin{figure}
         \newcommand{\sizefield}{7}
		\begin{tikzpicture}[scale=0.3]
			\foreach \x in {0, 1, ..., \sizefield} {
				\draw (\x, 0) -- (\x, \sizefield);
				\draw (0, \x) -- (\sizefield, \x);
			}
			\node at (0.5, 0.5) {\includegraphics[width=0.3cm]{images/E097_color.png}};%power
			\node at (6.5, 6.5) {\includegraphics[width=0.3cm]{images/1F4A7_color.png}};%water
			\node at (4.5, 3.5) {\includegraphics[width=0.3cm]{images/E1D2_color.png}};%drone
			\node at (3.5, 3.5) {$\obsright$};
		\end{tikzpicture}
        \begin{tikzpicture}[yscale=1.3]
			\node[world] (s) {$water$};
			\node at (-0.5, -0.4) {\small$\expwater$};
			\node[world] (t) at (4, 0) {$power$};
			\node at (4.5, -0.4) {\small$\exppower$};
			\node[world] (u) at (0, 1) {$patrol$};
			\node at (2.3, 1) {\small$\exppatrol$};
			\node[left = 0mm of s] {$s$};
			\node[right = 0mm of t] {$t$};
			\node[left = 0mm of u] {$u$};
			\draw (s) edge node[above] {$A,B$} (t);
			\draw (s) edge node[left] {$A$} (u);
			\draw (t) edge node[above] {$A$} (u);
% 		\end{tikzpicture}
% 		\begin{tikzpicture}
% 			\node at (8, 1.5) {$\obsleft$: one step left};
% 			\node at (8, 1) {$\obsright$: one step right};
% 			\node at (8, 0.5) {$\obsup$: one step up};
% 			\node at (8, 0) {$\obsdown$: one step down};
		\end{tikzpicture}
    \end{figure}
    \footnotetext[1]{\tiny H van Ditmarsch, S Ghosh,
R Verbrugge, and Y Wang. Hidden protocols: Modifying
our expectations in an evolving world. Artificial Intelligence, 208:18–40, 2014}
\end{frame}

\begin{frame}{Epistemic Expectation Model\footnotemark[1]}
    \begin{itemize}
         \item Epistemic model (W, R, V).
        \item Each world is assigned with a regular expression: a set of expected observations.
        % \item Each expression is expected observations in that particular possible world.
        \item Model can get truncated after a sequence of observation, say $\obsright$.
    \end{itemize}
    \begin{figure}
        \newcommand{\sizefield}{7}
		\begin{tikzpicture}[scale=0.3]
			\foreach \x in {0, 1, ..., \sizefield} {
				\draw (\x, 0) -- (\x, \sizefield);
				\draw (0, \x) -- (\sizefield, \x);
			}
			\node at (0.5, 0.5) {\includegraphics[width=0.3cm]{images/E097_color.png}};%power
			\node at (6.5, 6.5) {\includegraphics[width=0.3cm]{images/1F4A7_color.png}};%water
			\node at (4.5, 3.5) {\includegraphics[width=0.3cm]{images/E1D2_color.png}};%drone
			\node at (3.5, 3.5) {$\obsright$};
		\end{tikzpicture}
        \begin{tikzpicture}[yscale=1.3]
			\node[world] (s) {$water$};
			\node at (-0.5, -0.4) {\small$\expwater$};
			\node[world] (t) at (4, 0) {$power$};
			\node at (4.5, -0.4) {\small$(\obsleft \union \obsdown)^*$\textcolor{white}{$(\obsup \union \obsright \union \ep) (\obsleft \union \obsdown)^*$}};
			\node[world] (u) at (0, 1) {$patrol$};
			\node at (2.9, 1) {\small$\obsright^*\obsdown^+\obsleft^+\exppatrol$};
			\node[left = 0mm of s] {$s$};
			\node[right = 0mm of t] {$t$};
			\node[left = 0mm of u] {$u$};
			\draw (s) edge node[above] {$A,B$} (t);
			\draw (s) edge node[left] {$A$} (u);
			\draw (t) edge node[above] {$A$} (u);
% 		\end{tikzpicture}
% 		\begin{tikzpicture}
% 			\node at (8, 1.5) {$\obsleft$: one step left};
% 			\node at (8, 1) {$\obsright$: one step right};
% 			\node at (8, 0.5) {$\obsup$: one step up};
% 			\node at (8, 0) {$\obsdown$: one step down};
		\end{tikzpicture}
    \end{figure}
    \footnotetext[1]{\tiny H van Ditmarsch, S Ghosh,
R Verbrugge, and Y Wang. Hidden protocols: Modifying
our expectations in an evolving world. Artificial Intelligence, 208:18–40, 2014}
\end{frame}
 
\begin{frame}{Epistemic Expectation Model\footnotemark[1]}
    \begin{itemize}
         \item Epistemic model (W, R, V).
        \item Each world is assigned with a regular expression: a set of expected observations.
        % \item Each expression is expected observations in that particular possible world.
        \item Model can get truncated after a sequence of observation, say $\obsright\obsdown$.
    \end{itemize}
    \begin{figure}
        \newcommand{\sizefield}{7}
		\begin{tikzpicture}[scale=0.3]
			\foreach \x in {0, 1, ..., \sizefield} {
				\draw (\x, 0) -- (\x, \sizefield);
				\draw (0, \x) -- (\sizefield, \x);
			}
			\node at (0.5, 0.5) {\includegraphics[width=0.3cm]{images/E097_color.png}};%power
			\node at (6.5, 6.5) {\includegraphics[width=0.3cm]{images/1F4A7_color.png}};%water
			\node at (4.5, 2.5) {\includegraphics[width=0.3cm]{images/E1D2_color.png}};%drone
			\node at (3.5, 3.5) {$\obsright$};
			\node at (4.5, 3.5) {$\obsdown$};
		\end{tikzpicture}
        \begin{tikzpicture}[yscale=1.3]
			\node[world] (s) {$water$};
			\node at (-0.5, -0.4) {\small$\expwater$};
			\node[world] (t) at (4, 0) {$power$};
			\node at (4.5, -0.4) {\small$(\obsleft \union \obsdown)^*$\textcolor{white}{$(\obsup \union \obsright \union \ep) (\obsleft \union \obsdown)^*$}};
			\node[world] (u) at (0, 1) {$patrol$};
			\node at (2.9, 1) {\small$\obsright^*\obsdown^+\obsleft^+\exppatrol$};
			\node[left = 0mm of s] {$s$};
			\node[right = 0mm of t] {$t$};
			\node[left = 0mm of u] {$u$};
			\draw (s) edge node[above] {$A,B$} (t);
			\draw (s) edge node[left] {$A$} (u);
			\draw (t) edge node[above] {$A$} (u);
% 		\end{tikzpicture}
% 		\begin{tikzpicture}
% 			\node at (8, 1.5) {$\obsleft$: one step left};
% 			\node at (8, 1) {$\obsright$: one step right};
% 			\node at (8, 0.5) {$\obsup$: one step up};
% 			\node at (8, 0) {$\obsdown$: one step down};
		\end{tikzpicture}
    \end{figure}
    \footnotetext[1]{\tiny H van Ditmarsch, S Ghosh,
R Verbrugge, and Y Wang. Hidden protocols: Modifying
our expectations in an evolving world. Artificial Intelligence, 208:18–40, 2014}
\end{frame}

\begin{frame}{Epistemic Expectation Model\footnotemark[1]}
    \begin{itemize}
         \item Epistemic model (W, R, V).
        \item Each world is assigned with a regular expression: a set of expected observations.
        % \item Each expression is expected observations in that particular possible world.
        \item Model can get truncated after a sequence of observation, say $\obsright\obsdown$.
    \end{itemize}
    \begin{figure}
        \newcommand{\sizefield}{7}
		\begin{tikzpicture}[scale=0.3]
			\foreach \x in {0, 1, ..., \sizefield} {
				\draw (\x, 0) -- (\x, \sizefield);
				\draw (0, \x) -- (\sizefield, \x);
			}
			\node at (0.5, 0.5) {\includegraphics[width=0.3cm]{images/E097_color.png}};%power
			\node at (6.5, 6.5) {\includegraphics[width=0.3cm]{images/1F4A7_color.png}};%water
			\node at (4.5, 2.5) {\includegraphics[width=0.3cm]{images/E1D2_color.png}};%drone
			\node at (3.5, 3.5) {$\obsright$};
			\node at (4.5, 3.5) {$\obsdown$};
		\end{tikzpicture}
        \begin{tikzpicture}[yscale=1.3]
			\node[world] (s) {$water$};
			\node at (-0.5, -0.4) {\small\textcolor{white}{$(\obsright \union \obsup)^* (\obsdown \union \obsleft \union \ep)$}$(\obsright \union \obsup)^*$};
			\node[world] (t) at (4, 0) {$power$};
			\node at (4.5, -0.4) {\small$(\obsleft \union \obsdown)^*$\textcolor{white}{$(\obsup \union \obsright \union \ep) (\obsleft \union \obsdown)^*$}};
			\node[world] (u) at (0, 1) {$patrol$};
			\node at (2.5, 1) {\small$\obsdown^*\obsleft^+\exppatrol$};
			\node[left = 0mm of s] {$s$};
			\node[right = 0mm of t] {$t$};
			\node[left = 0mm of u] {$u$};
			\draw (s) edge node[above] {$A,B$} (t);
			\draw (s) edge node[left] {$A$} (u);
			\draw (t) edge node[above] {$A$} (u);
% 		\end{tikzpicture}
% 		\begin{tikzpicture}
% 			\node at (8, 1.5) {$\obsleft$: one step left};
% 			\node at (8, 1) {$\obsright$: one step right};
% 			\node at (8, 0.5) {$\obsup$: one step up};
% 			\node at (8, 0) {$\obsdown$: one step down};
		\end{tikzpicture}
    \end{figure}
    \footnotetext[1]{\tiny H van Ditmarsch, S Ghosh,
R Verbrugge, and Y Wang. Hidden protocols: Modifying
our expectations in an evolving world. Artificial Intelligence, 208:18–40, 2014}
\end{frame}

\begin{frame}{Epistemic Expectation Model\footnotemark[1]}
    \begin{itemize}
         \item Epistemic model (W, R, V).
        \item Each world is assigned with a regular expression: a set of expected observations.
        % \item Each expression is expected observations in that particular possible world.
        \item Model can get truncated after a sequence of observation, say $\obsright\obsdown\obsleft$.
    \end{itemize}
    \begin{figure}
        \newcommand{\sizefield}{7}
		\begin{tikzpicture}[scale=0.3]
			\foreach \x in {0, 1, ..., \sizefield} {
				\draw (\x, 0) -- (\x, \sizefield);
				\draw (0, \x) -- (\sizefield, \x);
			}
			\node at (0.5, 0.5) {\includegraphics[width=0.3cm]{images/E097_color.png}};%power
			\node at (6.5, 6.5) {\includegraphics[width=0.3cm]{images/1F4A7_color.png}};%water
			\node at (3.5, 2.5) {\includegraphics[width=0.3cm]{images/E1D2_color.png}};%drone
			\node at (3.5, 3.5) {$\obsright$};
			\node at (4.5, 3.5) {$\obsdown$};
			\node at (4.5, 2.5) {$\obsleft$};
		\end{tikzpicture}
        \begin{tikzpicture}[yscale=1.3]
			\node[world] (s) {$water$};
			\node at (-0.5, -0.4) {\small\textcolor{white}{$(\obsright \union \obsup)^* (\obsdown \union \obsleft \union \ep)$}$(\obsright \union \obsup)^*$};
			\node[world] (t) at (4, 0) {$power$};
			\node at (4.5, -0.4) {\small$(\obsleft \union \obsdown)^*$\textcolor{white}{$(\obsup \union \obsright \union \ep) (\obsleft \union \obsdown)^*$}};
			\node[world] (u) at (0, 1) {$patrol$};
			\node at (2.5, 1) {\small$\obsdown^*\obsleft^+\exppatrol$};
			\node[left = 0mm of s] {$s$};
			\node[right = 0mm of t] {$t$};
			\node[left = 0mm of u] {$u$};
			\draw (s) edge node[above] {$A,B$} (t);
			\draw (s) edge node[left] {$A$} (u);
			\draw (t) edge node[above] {$A$} (u);
% 		\end{tikzpicture}
% 		\begin{tikzpicture}
% 			\node at (8, 1.5) {$\obsleft$: one step left};
% 			\node at (8, 1) {$\obsright$: one step right};
% 			\node at (8, 0.5) {$\obsup$: one step up};
% 			\node at (8, 0) {$\obsdown$: one step down};
		\end{tikzpicture}
    \end{figure}
    \footnotetext[1]{\tiny H van Ditmarsch, S Ghosh,
R Verbrugge, and Y Wang. Hidden protocols: Modifying
our expectations in an evolving world. Artificial Intelligence, 208:18–40, 2014}
\end{frame}

\begin{frame}{Epistemic Expectation Model\footnotemark[1]}
    \begin{itemize}
         \item Epistemic model (W, R, V).
        \item Each world is assigned with a regular expression: a set of expected observations.
        % \item Each expression is expected observations in that particular possible world.
        \item Model can get truncated after a sequence of observation, say $\obsright\obsdown\obsleft$.
    \end{itemize}
    \begin{figure}
        \newcommand{\sizefield}{7}
		\begin{tikzpicture}[scale=0.3]
			\foreach \x in {0, 1, ..., \sizefield} {
				\draw (\x, 0) -- (\x, \sizefield);
				\draw (0, \x) -- (\sizefield, \x);
			}
			\node at (0.5, 0.5) {\includegraphics[width=0.3cm]{images/E097_color.png}};%power
			\node at (6.5, 6.5) {\includegraphics[width=0.3cm]{images/1F4A7_color.png}};%water
			\node at (3.5, 2.5) {\includegraphics[width=0.3cm]{images/E1D2_color.png}};%drone
			\node at (3.5, 3.5) {$\obsright$};
			\node at (4.5, 3.5) {$\obsdown$};
			\node at (4.5, 2.5) {$\obsleft$};
		\end{tikzpicture}
        \begin{tikzpicture}[yscale=1.3]
			\node[worldwhite] (s) {\textcolor{white}{$water$}};
			\node at (-0.5, -0.4) {\small\textcolor{white}{$(\obsright \union \obsup)^* (\obsdown \union \obsleft \union \ep)(\obsright \union \obsup)^*$}};
			\node[world] (t) at (4, 0) {$power$};
			\node at (4.5, -0.4) {\small$(\obsleft \union \obsdown)^*$\textcolor{white}{$(\obsup \union \obsright \union \ep) (\obsleft \union \obsdown)^*$}};
			\node[world] (u) at (0, 1) {$patrol$};
			\node at (2.5, 1) {\small$\obsleft^*\exppatrol$};
% 			\node[left = 0mm of s] {$s$};
			\node[right = 0mm of t] {$t$};
			\node[left = 0mm of u] {$u$};
% 			\draw (s) edge node[above] {$A,B$} (t);
% 			\draw (s) edge node[left] {$A$} (u);
			\draw (t) edge node[above] {$A$} (u);
% 		\end{tikzpicture}
% 		\begin{tikzpicture}
% 			\node at (8, 1.5) {$\obsleft$: one step left};
% 			\node at (8, 1) {$\obsright$: one step right};
% 			\node at (8, 0.5) {$\obsup$: one step up};
% 			\node at (8, 0) {$\obsdown$: one step down};
		\end{tikzpicture}
    \end{figure}
    \footnotetext[1]{H van Ditmarsch, S Ghosh,
R Verbrugge, and Y Wang. Hidden protocols: Modifying
our expectations in an evolving world. Artificial Intelligence, 208:18–40, 2014}
\end{frame}
 
%  \begin{frame}{Epistemic Expectation Model\footnotemark[1]}
%   \begin{itemize}
%       \item Epistemic model (W, R, V).
%         \item Each world is assigned with a regular expression: a set of expected observations.
%         % \item Each expression is expected observations in that particular possible world.
%         \item Model can get truncated after an observation, say $\obsright\obsdown\obsleft$.
%     \end{itemize}
%     \begin{figure}
%          \begin{tikzpicture}[yscale=1.3]
% % 			\node[world] (s) {$water$};
% % 			\node at (-0.5, -0.4) {$\small\expwater$};
% 			\node[world] (t) at (4, 0) {$power$};
% 			\node at (4.5, -0.4) {\small$(\obsleft\union\obsdown)^*$};
% 			\node[world] (u) at (0, 1) {$patrol$};
% 			\node at (2.7, 1) {\small$\obsleft^*\obsup^+\exppatrol$};
% % 			\node[left = 0mm of s] {$s$};
% 			\node[right = 0mm of t] {$t$};
% 			\node[left = 0mm of u] {$u$};
% % 			\draw (s) edge node[above] {$A,B$} (t);
% % 			\draw (s) edge node[left] {$A$} (u);
% 			\draw (t) edge node[above] {$B$} (u);
% 			\node at (8, 1.5) {$\obsleft$: one step left};
% 			\node at (8, 1) {$\obsright$: one step right};
% 			\node at (8, 0.5) {$\obsup$: one step up};
% 			\node at (8, 0) {$\obsdown$: one step down};
% 		\end{tikzpicture}
%     \end{figure}
%     \footnotetext[1]{\tiny H van Ditmarsch, S Ghosh,
% R Verbrugge, and Y Wang. Hidden protocols: Modifying
% our expectations in an evolving world. Artificial Intelli-
% gence, 208:18–40, 2014}
% \end{frame}
 
\begin{frame}{Public Observation Logic: Syntax\footnotemark[1]}
The language of $\POL$:
\vspace{.1cm}
			
		     $\begin{array}{r@{\quad::= \quad}l}
				\phi  &
				\top
				\mid
				p
				\mid \neg \phi
				\mid \phi \land \phi
				\mid K_i\phi
				\mid \hat{K}_i \phi
				\mid [\pi] \phi
				\mid \ldiaarg{\pi} \phi
				% 	   \mid [!\pi]\phi
				% \pi  &
				%\ep\mid
				% \dl\mid	
				%a
				% 	  \mid ?\phi_b
				%\mid \pi\cdot \pi
				% \mid \pi + \pi
				% \mid \pi^*\\
			\end{array}$
			
			\vspace{.1cm}
    \begin{itemize}
        \item<2-> $K_i\varphi$: an agent $i$ \textbf{knows} $\varphi$ holds.
        \vspace{.1cm}
        \begin{itemize}
            \item<3-> $\hat{K}_i\phi$: an agent $i$ considers $\varphi$ \textbf{possibly} holds.
        \end{itemize}
        \vspace{.1cm}
        \item<4-> $[\pi]\varphi$: after any sequence of \textbf{observation} matching $\pi$, $\varphi$ holds.
        \vspace{.1cm}
        \begin{itemize}
            \item<5-> $\ldiaarg{\pi}\phi$: after some sequence of \textbf{observation} matching $\pi$, $\varphi$ holds.
        \end{itemize}
        \vspace{.1cm}
        \item<6-> For example, $[\obsright\obsdown\obsleft] K_A \neg \textrm{ water }$
    \end{itemize}
    \footnotetext[1]{\tiny H van Ditmarsch, S Ghosh,
R Verbrugge, and Y Wang. Hidden protocols: Modifying
our expectations in an evolving world. Artificial Intelligence, 208:18–40, 2014}
\end{frame}

% \begin{frame}{Brief Background}
%     \begin{itemize}
%         \item The Epistemic Model (possible world) was in the works of Kripke.
        
%         \item $\POL$ and The Expectaction Model was by van Ditmarsch et.al.\footnotemark[1]
%     \end{itemize}
%     \footnotetext[1]{\tiny H van Ditmarsch, S Ghosh,
% R Verbrugge, and Y Wang. Hidden protocols: Modifying
% our expectations in an evolving world. Artificial Intelligence, 208:18–40, 2014}
% \end{frame}




% \begin{frame}{Recall the Questions}
%    Recall these questions:\pause
%     \begin{itemize}
%     \setlength\itemsep{1em}
%         \item<2-> Two kind of questions using Linear Programs:
%         \begin{itemize}
%         \setlength\itemsep{1em}
%             \item<3-> Given \textbf{values of the variables} in a \textbf{Linear Programming instance}, is it a feasible solution of the instance?:\textcolor{red}{Checking an assignment in a CNF}

%             \item<4-> Given a \textbf{Linear Program instance}, does it have a feasible solution?:\textcolor{red}{Checking whether there is an assignment of a CNF}
%         \end{itemize}

%         \item<5-> Two Questions involving OS programs:
%         \begin{itemize}
%         \setlength\itemsep{1em}
%             \item<6-> Given \textbf{an OS source code}, does it eventually arrive at deadlock?:\textcolor{red}{Verify with respect to the finite program}

%             \item<7-> Given my \textbf{specifications}, such as "the program should never arrive at a deadlock", etc, can the required program be synthesised?:\textcolor{red}{Check whether there is a program}
%         \end{itemize}
%     \end{itemize}
% \end{frame}

\section{Previous Contribution}

\begin{frame}{The Model Checking Question}
    \begin{figure}
        \centering
        \begin{tikzpicture}
            \node (rect) at (4,2) [draw,thick,minimum width=2cm,minimum height=2cm] {$\POL$ Model-check};
            \node(M)[rectNode, left of=rect, xshift = -3cm]{$\POL$ Model $\M$};
            \node(s)[rectNode, above of=M]{Possible world $s$};
            \node(f)[rectNode, below of=M]{$\POL$ formula $\varphi$};
            \node(op)[rectNode, right of=rect, xshift = 3cm]{$\M,s\vDash\varphi$?};
            \draw[->](s.east)--node[anchor=south] {}(rect.west);
            \draw[->](M.east)--node[anchor=south] {}(rect.west);
            \draw[->](f.east)--node[anchor=south] {}(rect.west);
            \draw[->](rect.east)--node[anchor=south] {}(op.west);
        \end{tikzpicture}
    \end{figure}
   \begin{block}{Theorem: $\POL$ Model Checking Complexity [IJCAI'22]}
    The model-checking problem of $\POL$ is $\PSPACE$-Complete.
    \end{block}
\end{frame}

\begin{frame}{$\POL$ Model-checking}
    Recall the example:\\
    \begin{figure}
		\newcommand{\sizefield}{7}
		\begin{tikzpicture}[scale=0.3]
			\foreach \x in {0, 1, ..., \sizefield} {
				\draw (\x, 0) -- (\x, \sizefield);
				\draw (0, \x) -- (\sizefield, \x);
			}
			\node at (0.5, 0.5) {\includegraphics[width=0.3cm]{images/E097_color.png}};%power
			\node at (6.5, 6.5) {\includegraphics[width=0.3cm]{images/1F4A7_color.png}};%water
			\node at (3.5, 3.5) {\includegraphics[width=0.3cm]{images/E1D2_color.png}};%drone
		\end{tikzpicture}
		\begin{tikzpicture}[yscale=1.3]
			\node[world] (s) {\fbox{$water$}};
			\node at (-0.1, -0.4) {\small $\expwater$};
			\node[world] (t) at (4, 0) {$power$};
			\node at (4.3, -0.4) {\small $\exppower$};
			\node[world] (u) at (0, 1) {$patrol$};
			\node at (2.3, 1) {\small $\exppatrol$};
			\node[left = 0mm of s] {$s$};
			\node[right = 0mm of t] {$t$};
			\node[left = 0mm of u] {$u$};
			\draw (s) edge node[above] {$A,B$} (t);
			\draw (s) edge node[left] {$A$} (u);
			\draw (t) edge node[above] {$A$} (u);
		\end{tikzpicture}
    \end{figure}\pause
    \textbf{Question:} Does there exist a sequence of \raisebox{-1.5
    mm}{\drone}-moves \pause or a \textbf{PLAN}\pause  
  
  
  after which Knowledge of an agent changes? (Epistemic Planning\footnotemark[1])\pause

  
    \textbf{Solution:} $\M,s\vDash\ldiaarg{(\obsright\union\obsdown\union\obsleft\union\obsup)^\star}K_A\varphi$\pause~ (\textcolor{Fuchsia}{\textbf{Model-Checking}})
    \footnotetext[1]{\tiny T. Bolander, A Gentle Introduction to Epistemic Planning: The DEL Approach, M4M@ICLA 2017}
    
\end{frame}


% \begin{frame}{$\POL$ Model-checking}
%         \begin{figure}
% 		\newcommand{\sizefield}{7}
% 		\begin{tikzpicture}[scale=0.3]
% 			\foreach \x in {0, 1, ..., \sizefield} {
% 				\draw (\x, 0) -- (\x, \sizefield);
% 				\draw (0, \x) -- (\sizefield, \x);
% 			}
% 			\node at (0.5, 0.5) {\includegraphics[width=0.3cm]{images/E097_color.png}};%power
% 			\node at (6.5, 6.5) {\includegraphics[width=0.3cm]{images/1F4A7_color.png}};%water
% 			\node at (3.5, 3.5) {\includegraphics[width=0.3cm]{images/E1D2_color.png}};%drone
% 		\end{tikzpicture}
% 		\begin{tikzpicture}[yscale=1.3]
% 			\node[world] (s) {\fbox{$water$}};
% 			\node at (-0.1, -0.4) {\small $\expwater$};
% 			\node[world] (t) at (4, 0) {$power$};
% 			\node at (4.3, -0.4) {\small $\exppower$};
% 			\node[world] (u) at (0, 1) {$patrol$};
% 			\node at (2.3, 1) {\small $\exppatrol$};
% 			\node[left = 0mm of s] {$s$};
% 			\node[right = 0mm of t] {$t$};
% 			\node[left = 0mm of u] {$u$};
% 			\draw (s) edge node[above] {$A,B$} (t);
% 			\draw (s) edge node[left] {$A$} (u);
% 			\draw (t) edge node[above] {$A$} (u);
% 		\end{tikzpicture}
%     \end{figure}
%     \begin{itemize}
%         \item \textbf{Epistemic planning, $\existential$ Fragment}: only $\ldiaarg{}$, \textbf{not } $[]$.
% $$\M, s \models \ldiaarg{(\obsright \union \obsdown \union \obsleft \union \obsup)^*} (K_B water \land \hat K_A patrolling)$$\pause
% Does there exist \textbf{a sequence of move} (a plan) such that after observation:
% \begin{itemize}
%     \item<3-> $B$ knows goal is water, but
%     \item<4-> $A$ still considers patrolling a possibility?
% \end{itemize}
%     \end{itemize}
% \end{frame}


% \begin{frame}{$\POL$ Model-checking}
%         \begin{figure}
% 		\newcommand{\sizefield}{7}
% 		\begin{tikzpicture}[scale=0.3]
% 			\foreach \x in {0, 1, ..., \sizefield} {
% 				\draw (\x, 0) -- (\x, \sizefield);
% 				\draw (0, \x) -- (\sizefield, \x);
% 			}
% 			\node at (0.5, 0.5) {\includegraphics[width=0.3cm]{images/E097_color.png}};%power
% 			\node at (6.5, 6.5) {\includegraphics[width=0.3cm]{images/1F4A7_color.png}};%water
% 			\node at (3.5, 3.5) {\includegraphics[width=0.3cm]{images/E1D2_color.png}};%drone
% 		\end{tikzpicture}
% 		\begin{tikzpicture}[yscale=1.3]
% 			\node[world] (s) {\fbox{$water$}};
% 			\node at (-0.1, -0.4) {\small $\expwater$};
% 			\node[world] (t) at (4, 0) {$power$};
% 			\node at (4.3, -0.4) {\small $\exppower$};
% 			\node[world] (u) at (0, 1) {$patrol$};
% 			\node at (2.3, 1) {\small $\exppatrol$};
% 			\node[left = 0mm of s] {$s$};
% 			\node[right = 0mm of t] {$t$};
% 			\node[left = 0mm of u] {$u$};
% 			\draw (s) edge node[above] {$A,B$} (t);
% 			\draw (s) edge node[left] {$A$} (u);
% 			\draw (t) edge node[above] {$A$} (u);
% 		\end{tikzpicture}
%     \end{figure}
%     \begin{itemize}
%         \item \textbf{Bounded Epistemic planning, $\starfree$-$\existential$ Fragment}: no $[]$ operator and no $*$ in $\pi$.
% $$\M, s \models \ldiaarg{(\obsright \union \obsdown \union \obsleft \union \obsup \union \epsilon)^4} (K_b water \land \hat K_a patrolling)$$\pause
% Does there exist a sequence of move \textbf{of length at most 4} (a bounded-length plan) such that after observation:
% \begin{itemize}
%     \item<3-> $B$ knows goal is water, but
%     \item<4-> $A$ still considers patrolling a possibility?
% \end{itemize}
%     \end{itemize}
% \end{frame}

\section{Our Current Contribution}

\begin{frame}{$\POL$ Satisfiability}
    Model Checker is a powerful tool, when the scenario is modeled.
    
    \pause
    \textbf{Question:} What about properties of multiple models?
    
    \pause
    \textbf{Question:} Is a certain \textcolor{red}{property} \textit{satisfied} in EVERY model having certain other \textcolor{Fuchsia}{properties} in common?\pause

    \textbf{Approach:} Specifying the models using the \textcolor{Fuchsia}{formulas}.\pause
    
    \textbf{For example:} 
    
   %  $\ldiaarg{\obsright}\top$\pause~~~~~~~~~~~~~~~~~~\textcolor{red}{$\rightarrow$}~~~~~ \raisebox{-2mm}{\begin{tikzpicture}
   %      \node[world] (s) {\fbox{~}};
			% \node[right of = s, xshift = -0.5cm]{$\obsright$};
   %  \end{tikzpicture}}

    \textcolor{Fuchsia}{$water \wedge\hat{K}_A power$}\pause~~~\textcolor{red}{$\rightarrow$}~~~~~\raisebox{-2mm}{
    \begin{tikzpicture}
        \node[world] (s) {\fbox{$water$}};
        \node[above of = s, yshift = -0.5cm]{$\ep$};
	\node[world] (t) at (4, 0) {$power$};
        \node[right of = t]{$\ep$};
        \draw (s) edge node[above] {$A$} (t);
    \end{tikzpicture}},\\
    \pause~~~~~~~~~~~~~~~~~~~~~~~~~~~~~~~~~\raisebox{-2mm}{
    \begin{tikzpicture}
        \node[world] (s) {\fbox{$water, power$}};
        \node[above of = s, yshift = -0.5cm]{$\ep$};
	% \node[world] (t) at (4, 0) {$power$};
 %        \node[right of = t]{$\ep$};
 %        \draw (s) edge node[above] {$A$} (t);
    \end{tikzpicture}}
\end{frame}

\begin{frame}{POL Satisfiability}
    \textbf{Question:} Is certain \textcolor{red}{property} \textit{satisfied} in EVERY model having certain other \textcolor{Fuchsia}{properties} in common?\pause

    \textbf{Approach:} Specifying the models using the \textcolor{Fuchsia}{formulas} : \textcolor{Fuchsia}{$\varphi_M$}\pause

    ~~~~~~~~~~~~~~~Specify the \textcolor{red}{property} to verify: \textcolor{red}{$\varphi$}\pause

    ~~~~~~~~~~~~~~~Verify whether $\neg($\textcolor{Fuchsia}{$\varphi_M$}$\rightarrow$\textcolor{red}{$\varphi$}$)$ has a model\pause

    ~~~~~~~~~~~~~~~If \textcolor{red}{NO MODEL} then \textcolor{red}{$\varphi$} can hold \pause\textbf{without conflicting} \textcolor{Fuchsia}{$\varphi_M$}.

    
\end{frame}



\newcommand{\NEXPTIME}{\mathsf{NEXPTIME}}

\begin{frame}{$\POL$ Satisfiability Results}
     \begin{figure}
        \centering
        \begin{tikzpicture}
            \node (rect) at (4,2) [draw,thick,minimum width=2cm,minimum height=2cm] {$\POL$ SAT Checker};
            % \node(M)[rectNode, left of=rect, xshift = -3cm]{$\POL$ Model $\M$};
            % \node(s)[rectNode, above of=M]{Possible world $s$};
            \node(f)[rectNode, left of=rect, xshift = -3cm]{$\POL$ formula $\varphi$};
            \node(op)[rectNode, right of=rect, xshift = 3cm]{$\exists ? \M,s :\M,s\vDash\varphi$?};
            % \draw[->](s.east)--node[anchor=south] {}(rect.west);
            % \draw[->](M.east)--node[anchor=south] {}(rect.west);
            \draw[->](f.east)--node[anchor=south] {}(rect.west);
            \draw[->](rect.east)--node[anchor=south] {}(op.west);
        \end{tikzpicture}
    \end{figure}\pause
    Is $\POL$-Sat \textbf{decidable}?\\\pause
    Without Kleene Star ($\POL^-$): \textbf{Yes}.
    \begin{block}{$\POL$ Starfree Satisfiability Complexity}
        The satisfiability problem of $\POL^-$ is $\NEXPTIME$-Complete.
    \end{block}\pause

    \textbf{Upper Bound:} Sound and Complete Tableau System
\end{frame}

\begin{frame}{Upper Bound $\POL^-$: Tableau Term}
    \begin{itemize}
    \setlength\itemsep{4em}
        \item<1-> \textbf{Survival Term}
        \begin{itemize}
            \item<2-> \textbf{$(\sigma~~~w~~~\checkmark)$}: World $\sigma$ survives after updated by $w$.
        \end{itemize}

        \item<3-> \textbf{Formula Term}
        \begin{itemize}
            \item<4->  \textbf{$(\sigma~~~w~~~\varphi)$}: $\varphi$ holds in World $\sigma$ after updating by $w$ 
        \end{itemize}
    \end{itemize}
\end{frame}

\newcommand{\tableaurulesection}[1]{\\ \multicolumn{2}{c}{\textbf{#1}} \\}
\newcommand{\tableaurule}[2]{\raisebox{2.5mm}{#1} & #2 \\[2mm]}


\begin{frame}{Upper Bound $\POL^-$: Some Tableau Rules}
    \begin{figure}[t!]
	\scalebox{0.9}{
\begin{tabular}{lc}
    \multicolumn{2}{c}{\textbf{Propositional Rules}}\\
    \tableaurule{Clash rule}{
        \infer{\bot}{%
                        (\sigma
                        & w
                        & p),
                        & (\sigma
                        & w
                        & \neg p)
                        }
                    }
    
    \tableaurulesection{Diamond and Box Rules}
    
 
    \tableaurule{ Diamond Project}{
    \infer{(\sigma\ \ wa\ \ \checkmark), (\sigma\ \ wa\ \ \psi)}{%
    	(\sigma
    	& w
    	& \langle a\rangle\psi)
    }}
    
    \tableaurule{ Box Project}{
    \infer{(\sigma\ \ wa\ \ [\pi\regdiv a]\psi)}{%
    	(\sigma
    	& w
    	& [\pi]\psi),
    	& (\sigma\ \ wa\ \ \checkmark)
    }}
    
    
    
    
    \tableaurulesection{Survival Rules}
    
    
    \tableaurule{Constant Valuation Up}{
    \infer{(\sigma\ \ \epsilon\ \ p)}{%
    	(\sigma
    	& w
    	& p)
    }~~~~~
    \infer{(\sigma\ \ \epsilon\ \ \neg p)}{%
    	(\sigma
    	& w
    	& \neg p)
    }%, $p\in\mathcal{P}$
}
    
    \tableaurule{Survival Chain}
    {    \infer{(\sigma\ \ w\ \ \checkmark)}{%
    	(\sigma
    	& wa
    	& \checkmark)
    }
%, $a\in\Sigma, w\in\Sigma^*$
}



\end{tabular}
}
\end{figure}
\end{frame}

\newcommand{\PAL}{\mathsf{PAL}}
\begin{frame}{$\POL$ Satisfiability Results}
    \begin{center}
        \begin{tabular}{ m{9em} m{4cm} m{3cm} } 
         \textcolor{Fuchsia}{\textbf{$\starfree$} Multi-agent fragment} & \textcolor{Fuchsia}{$[aab+b]K_i p$}, \textcolor{Red}{\st{$[aab^*]K_i p$}} & $\NEXPTIME$-Complete \\
         & & ($2^k$-Tiling Problem)\\
         & &\\
         \textcolor{Fuchsia}{\textbf{$\starfree$} Single-agent fragment} & \textcolor{Fuchsia}{$[aab]K_i p\vee K_i q$}, \textcolor{Red}{\st{$[aab^*+b]K_i p\vee K_j q$}} & $\PSPACE$-Hard \\ 
         & & (TQBF)\\
         & &\\
         \textcolor{Fuchsia}{$\word$ Multi-agent fragment} & \textcolor{Fuchsia}{$[aab]K_i p$}, \textcolor{Red}{\st{$[aab+b]K_i p$}} & $\PSPACE$-Complete\\
         & & ($\PAL$ Reduction)\\
         & &\\
         \textcolor{Fuchsia}{$\word$ Single-agent fragment} & \textcolor{Fuchsia}{$[aab]K_i p\vee K_i q$}, \textcolor{Red}{\st{$[aab+b]K_i p\vee K_j q$}} & $\NP$-Complete\\
         & & ($\PAL$ Reduction)
        \end{tabular}
    \end{center}
\end{frame}





% \begin{frame}{Our Result}
%     \begin{block}{Theorem: $\POL$ Model Checking Complexity}
%     The model-checking problem of $\POL$ is $\PSPACE$-Complete.
%     \end{block}\pause
%     Too Hard?\\
%     Any easier fragment?
% \end{frame}

% \begin{frame}{Some Fragments of $\POL$}
%     Are there more efficient fragments?\pause
%     \begin{center}
%         \begin{tabular}{ m{15em} m{2cm} m{3cm} } 
%          (\textcolor{Fuchsia}{\textbf{$\starfree$} fragment}) No Kleene-stars in formula\pause & \textcolor{Fuchsia}{$[aab+b]K_i p$}, \textcolor{Red}{\st{$[aab^*]K_i p$}} & $\PSPACE$-Hard \\
%          & &\\
%          (\textcolor{red}{$\existential$ fragment}) Only existential operators in formula\pause & \textcolor{Fuchsia}{$\ldiaarg{aab^*}\hat{K}_i p$}, \textcolor{Red}{\st{$[aab^*]K_i p$}} & $\PSPACE$-Hard \\ \pause
%          & &\\
%          (\textcolor{red}{$\starfree-\existential$ fragment}) $\starfree$ fragment with $\existential$ constraint & \textcolor{Fuchsia}{$\ldiaarg{aab+b}\hat{K}_i p$}, \textcolor{Red}{\st{$\ldiaarg{aab^*}K_i p$}} & $\NP$-Complete \\ 
%          (\textcolor{red}{$\word$ fragment}) Only one observation in each regular modality & \textbf{:} & $\PTime$
%         \end{tabular}
%     \end{center}
% \end{frame}



% \begin{frame}{Some Fragments of $\POL$}
%     Any easier fragments?
%     \begin{itemize}
%         \item<1-> ($\starfree$ fragment) No Kleene-stars in public observations in formula: $\PSPACE$-Hard.
%         \item<2-> ($\existential$ fragment) Only existential operators in formula: $\PSPACE$-Hard
%         \item<3-> ($\starfree-\existential$ fragment) $\starfree$ fragment with $\existential$ constraint: $\NP$-Complete.
%         \item<4-> ($\word$ fragment) Only one observation in each regular modality: $\PTime$.
%     \end{itemize}
% \end{frame}



\section{Conclusion}

\begin{frame}{Concluding...}
    \begin{itemize}
    \item  Model-Checking and Satisfiability problem for $\POL$ (Full language ongoing)
 
     \item Complete Axiomatic System for extension of $\POL$ : Epistemic Protocol Logic\footnotemark[1]

     \item Programs can be interpreted more efficiently in CFL.

     How about CFG instead of regular?
 \end{itemize}
 \footnotetext[1]{\tiny H van Ditmarsch, S Ghosh,
R Verbrugge, and Y Wang. Hidden protocols: Modifying
our expectations in an evolving world. Artificial Intelligence, 208:18–40, 2014}
\end{frame}

 
\end{document}

\begin{frame}{Thank You}
    \begin{figure}
        \centering
        \includegraphics{images/question.jpg}
    \end{figure}
\end{frame}